\documentclass[ author={Stephen Livermore-Tozer},
				supervisor={Dr. Peter Flach},
				degree={MEng},
				title={Performing Algorithmic Co-composition Using Machine Learning},
				subtitle={},
				type={research},
				year={2016} ]{dissertation}



\begin{document}
	
	\maketitle
	
	\frontmatter
	
	\makedecl
	
	\tableofcontents
	\lstlistoflistings
	
	% -----------------------------------------------------------------------------
	
	\chapter*{Executive Summary}
	
	% Hypothesis
	For my research, I have investigated the hypothesis that existing methods of performing melodic composition using machine learning can be adapted to successfully perform co-composition with a human composer. In this context, success may be measured by subjective impression received from experts in composition. The exact set of tasks involved in the co-composition process is not fixed; part of this research involved finding the tasks that are both a) useful to the composer, and b) within the capabilities of current AI.
	
	% Achievements
	In the pursuit of this objective, I have completed the following tasks:
	\begin{itemize}
		\item I surveyed a number of expert composers to identify the tasks within the domain of composition whose partial automation would provide the greatest benefit to the composer.
		\item I performed research over existing machine learning methods for algorithmic composition, determining their applicability and estimated efficacy within the tasks obtained above.
		\item I implemented an application that may be trained on a set of music data (stored in the Midi format) and will provide an attempted continuation of a given musical input, according to the tasks obtained above.
		\item I designed and performed a series of objective and subjective tests to assess the performance of the application and the algorithms that comprise it.
	\end{itemize}
	
	
	% -----------------------------------------------------------------------------
	
	\chapter*{Supporting Technologies}
	
	I used the Anaconda implementation of Python to create my application. Furthermore, I used a set of libraries implementing machine learning techniques to create the application. These libraries are:
	
	\begin{itemize}
		\item NumPy for efficient mathematical operations
		\item SciPy for clustering and regression
		\item PyBrain for neural networks
		\item HmmLearn for hidden Markov models
	\end{itemize}
	
	% -----------------------------------------------------------------------------
	
	\chapter*{Acknowledgements}
	
	I would like to thank my supervisor Peter Flach for his guidance over the course of this project in both its technical aspects and in its effective management.
	
	I would also like to thank my good friend Thomas Norton, for his expert musical advice and support throughout my work.
	
	\mainmatter
	
	% -----------------------------------------------------------------------------
	
	\chapter{Contextual Background}
	\label{chap:context}
	
	\section{Algorithmic Composition}

	\subsection{Overview}
	
	Algorithmic composition is defined as the use of algorithms to compose, in whole or in part, a musical score. This is generally accomplished using a computer, although many earlier systems could be traced out by hand. There exist a large and diverse number of algorithms designed for this task, accompanied by many different representations for musical data. Among these algorithms are a large number of methods for learning musical creativity, including supervised learning, unsupervised learning, rules and constraints, and stochastic modelling, among others.
	
	Although many methods for algorithmic composition have been defined over the past several decades, few have achieved notably useful or successful results. Even relatively sophisticated and advanced methods tend to produce music with notably inhuman idiosyncrasies or that is subjectively described as ``bland.'' There are several key contributing factors to this problem, two of which are the lack of both computational power the unavailability of sufficient quantities of useful data. One way of describing the core issue however is that there does not currently exist a good way of measuring or analysing human subjective impression to a broad or deep enough scale. Because of this nearly all current methods either attempt to codify compositional principles as used by humans, extract features from music known to be ``good,'' or perform some combination of the two. This is not necessarily an incorrect approach, but knowledge directly inscribed by humans is universally insufficient to capture all the necessary complexity of music created by humans, and it is currently prohibitively difficult for machine learning to extract meaningful features from existing music.
	
	\subsection{Previous Work}
	
	Using algorithms or similar formal methods in the composition of music is not a purely digital phenomenon - certain forms in classical music, such as \textit{canon}, involved using a set of instructions to develop a simple melody into a whole segment of a composition. 
	% TODO: Some more examples of similar stuff
	
	The introduction of computers to the process of algorithmic composition occurred in the mid 1950s. At this time a number of experiments were proposed, and the first actual production of a computer-composed piece took place in 1957.
	
	\subsection{Rule-based systems}
	
	\section{Other stuff}
	
	% Current state of AI
	Artificial Intelligence in its current state is a major component of the current technological landscape. As a broad field, AI plays a significant role in an enormous number of areas with a large technological component, including large industries such as finance, medicine, marketing, and internet services. In particular one of the most frequent and powerful uses is to learn from large quantities of data and perform analytical tasks, such as classifying samples in some way or identifying complex trends. 
	
	% Background of algorithmic composition
	
	
	% What's the deal with it
	There exist several reasons for the interest in the creation of AI that can compose music. The most directly practical of these is that music is a multi-billion pound industry, and music production has been revolutionised by technology on many occasions. The power of introducing a significant degree of automation to the process may be immense on both a financial and cultural scale.
	
	Additionally, the application of AI to the arts is a heavily romanticized achievement; ``creativity'' is quite universally considered to be a shortcoming of AI, and the creation of art is generally considered to be one of the most significant expressions of human creativity. Because of this, AI that is capable of creating art comparable to that created by humans represents a technological and cultural milestone.
	
	Within the sphere of the arts, music is a key target due to its heavily pattern-based, mathematical nature. Even with these advantages however, AI performance in composition is generally considered to be distinctly poor. According to the majority of subjective measures, AI is identifiably so due to being lacking in creating both interesting short-term melodies and coherent long-term structure. 
	
	
	
	% -----------------------------------------------------------------------------
	
	\chapter{Technical Background}
	\label{chap:technical}
	
	\section{Overview}
	
	There are many different methods for performing algorithmic composition, with a broad range of algorithms and data structures used. Most major AI techniques have many corresponding applications to algorithmic composition - it is also quite common for these techniques to be combined, such as combining neural networks and evolutionary algorithms. % TODO: Link to section
	Additionally, many of these methods are focused on a specific subtask of composition; popular targets are chord identification, continuation of single line melody, and counterpoint, while many other algorithms focus on more specific tasks (typically relating to a particular genre, such as jazz or baroque).
	
	
	\section{Knowledge-Based Techniques}
	\label{sec:knowledge-techniques}
	
	Knowledge-based systems are those that use a symbolic representation of knowledge to complete a task, and were one of the first forms of AI created. This generally means storing a database of facts and then inferring with this data to resolve statements. In the case of algorithmic composition there are several common abstract approaches that use and represent knowledge in this way. 
	
	\subsection{Rule-Based Systems}
	
	The first attempts at digital algorithmic composition were knowledge-based, and typically involved encoding explicit, hand-written compositional rules into code and using those rules to create or modify music. These systems notably incorporated no learning of any kind - the behaviour of the program was entirely determined by the initial set of rules and the input (typically a randomly generated melody). This had the effect of making the systems very ``brittle,'' meaning that performance typically dropped dramatically whenever the system encountered an unideal input - a very common occurrence due to the limited scope covered by the initial rules. This can be attributed to the fact that while the compositional principles used by humans are useful and powerful in human hands, they are heavily dependent on the composer providing their own creative input throughout the process.
	
	\subsection{Musical Grammars}
	
	Other more computationally focused attempts at musical knowledge-based systems have made use of formal grammars. A grammar consists of a set of nonterminal symbols $N$, a set of terminal symbols $\Sigma$ (s.t. $N \cap \sigma = \emptyset$), a start symbol $S \in N$, and a set of production rules that transform strings containing at least 1 nonterminal symbol. A grammar can be used to produce strings by taking a string consisting of the start symbol and repeatedly applying valid production rules to the string until it contains only terminal symbols. Grammars are thus well suited to parsing and generating hierarchical structures recursively, which makes them a good fit for music. 
	
	Formal grammars for composition first began to gain popularity in the mid 1970s, with hand-crafted grammars used to represent the structure of music \cite{lidov1973melody}\cite{rader1974method}. 
	
	\section{Supervised Learning Techniques}
	\label{sec:supervised-techniques}
	
	Another major branch of work within the field is the use of supervised learning to train AIs through a set of examples, instead of encoding knowledge by hand. This does away with many of the disadvantages of knowledge-based systems, particularly their brittle nature - by learning from examples, the AI can feasibly adapt to work with any particular style of music (with varying difficulty). Currently supervised learning is a very popular method for algorithmic composition, with a majority of papers from the past decade relying on it in some fashion \cite{fernandez2013ai}. This is linked to a combination of the newly widespread availability of musical data to learn from, the increase in computational power (and hence potential complexity of learning models compared to static models), and the high popularity of supervised learning in general.
	
	Generally however these methods still have some significant drawbacks. One of the most fundamental of these flaws is the fact that due to their knowledge being entirely derived from a training set they are limited in their ability to develop novel material, instead simply imitating the styles observed during training. There have been attempts to overcome this issue, with many of the most successful attempts using a combination of different techniques (see \ref{sec:hybrid-techniques}).
	
	\subsection{Artificial Neural Networks}
	
	The most popular class of supervised learning techniques in the context of algorithmic composition are artificial neural networks (ANNs). This can be related to their ability to learn a mapping from one set of data to another, allowing them to be adapted to a complex problem domain such as music. The earliest use of neural networks for algorithmic composition was during the late 80s, where a simple 3-layer recurrent neural network (RNN) was used to compose short monophonic melodies \cite{todd1989connectionist}. Many subsequent ANNs have used a similar design, with adjustments made to the topology and data representation. 
	
	% -----------------------------------------------------------------------------
	
	\chapter{Project Execution}
	\label{chap:execution}
	
%	\section{Objectives}
%	
%	The primary goal of this research is to explore the potential of integrating modern AI methods into the process of human musical composition. Thus it was not only necessary to investigate the AI algorithms themselves, but also their place within a human composer's work environment. The justification for this step is clear - an AI capable of performing or assisting only with tasks that are already trivial adds very little utility. It is therefore necessary as a preliminary step to identify the specific compositional tasks that will receive the greatest positive impact from the introduction of automation.. 
	
	\section{Design}
	
	\subsection{Composition Algorithm}
	
	One of the key details that sets the intended method apart from most other algorithmic composition methods is its purely cooperative nature. This allows for a major performance benefit by offloading a certain portion the work to the human operator. There are three major places in which this offloading can take place: initializing state before generation, guiding the generation process, and adjusting the final generated output. The first is achieved by using the score written by the user as input to create new material from. The third is an intrinsic part of the process, as the final output of the algorithm is provided to the user to be used as they please. The second may be exploited to improve the performance of the algorithm in generating output that is useful or pleasing to the user.
	
	%Out of the wide range of algorithms used in algorithmic composition, the best suited to interactive composition of this variety are evolutionary algorithms (EAs). This is due to their use of a fitness function over a sample population; it is possible to augment the fitness assignment process by allowing the user to provide their preferences as an input to determine each sample's fitness. A number of EAs have been designed with this behaviour in mind in order to overcome the challenge of defining an objective fitness function. This method has the major drawback that it typically causes a large amount of fatigue to the user - providing a rating to each sample once for each generation is a prohibitively large task for a human. To overcome this, it is generally attempted to minimize the population size and number of generations presented to the user, often by querying the user for input over a small subset of fitness evaluations. 
	
	There are a number of ways that this effect could be achieved. One would be the use of evolutionary algorithms. Due to their use of a fitness function to guide their search through the solution space, it is simple and straightforward to integrate user feedback into the EA by using user rating - either as the fitness function, or as some kind of augmentation to the fitness assignment process. A number of EAs have been designed with this behaviour in mind in order to overcome the challenge of defining an objective fitness function. This does however suffer a major drawback in that it is not feasible for an individual user to guide the search process single-handedly; due to the large number of samples that must be judged, user fatigue rapidly becomes a critical constraint. There have been attempts to mitigate this effect through various means, such as reducing the population, number of generations, or novel methods such as using clustering to avoid presenting the user with similar outputs. These methods each present their own limitations however, to an extent that was prohibitive in the case of this research project. 

	% TODO: Explain and justify choice of algorithm									
	
	%\subsection{Musical Fitness}
	%\label{sec:fitness}
	
	%It is accurate to state that the EA is primarily defined by its fitness function: although choice of parameters and initial population may improve the rate at which it approaches an optimal solution, it is the fitness function that defines the optimal solutions. Because of this it is necessary to select a fitness function that will favour good results and prune bad ones. On an objective level this is completely infeasible if not impossible to achieve, hence the need for user intervention to guarantee a good result. Instead a more accurate (if quite abstract) goal for the fitness function should be to minimize the number of bad results presented to the user and adapt well to the user's direction. 
	
	%Through a review of the available literature I have determined that one of the best suited fitness functions to this task is the ARTMAP neural network fitness evaluator \cite{burton1998hybrid}. The main advantages to this method are its ability to adapt to a varied training set, its defined use of user input to direct the search process, and its inherent ability to detect similar patterns - thus preventing the user from being presented with the same results repeatedly. As a supervised learning technique it is necessary to provide a selection of training data
	
	%\section{Population Generation}
	%\label{sec:pop-gen}
	
	%The choice of generation algorithm for the initial population of an EA can have a major impact on that EA's overall performance. An EA with a heuristic that generates a good set of initial solutions can achieve an ideal solution several orders of magnitude faster than an EA that uses a random starting population. This effect is particularly enhanced in scenarios where the output domain is very large while the percentage of good solutions is very small, as is very much the case with musical scores. A simple heuristic for generating approximately feasible melodies does not exist however; in the domain of music finding an approximately good solution is often approximately as difficult as finding a good solution. It is therefore necessary to use an additional algorithmic composition method for the purposes of generating a starting population.
	
	%This method will need to satisfy several requirements to be viable as a population generator:
	
%	\begin{itemize}
%		\item It must have a high run-time speed, as it will need to generate a large number of possible solutions before the EA can begin.
%		\item It must be able to produce many different outputs for the same input, as the population must have variance for the search to be efficient.
%		\item It must be non-interactive in order to not add to the user's fatigue, due to the presence of fatigue as a constraining factor on the EA's performance.
%		\item It must build its solutions off of the score already produced by the user in order for its output to be relevant to the EA.
%	\end{itemize}
	
	%The first requirement implies the use of a technique that uses prior-obtained knowledge to find a solution rather than performing a broader search as with an EA. Two of the most effective and well-documented ways to achieve this are handcrafted knowledge-based systems (\ref{sec:knowledge-techniques}) and supervised learning (\ref{sec:supervised-techniques}). In the former case it is necessary to compose a set of rules and/or constraints that will provide good output for the user's choice of input; this is difficult as these types of techniques are known to be brittle and not very versatile.	In particular most effective rulesets only perform well for specific genres, which prohibits creating a generalist system. Supervised learning on the other hand has the advantage of being versatile and adaptable based on the type of data being worked with. It is also convenient in this case as it is possible to use data already obtained for training the EA's fitness function (\ref{sec:fitness}).
	
	%I reviewed a variety of supervised learning techniques that could be applied effectively to this task. The majority of documented methods use either recurrent neural networks or hidden Markov models trained on a set of melodies. The main advantage of these techniques with regard to composition is that they are effective at extracting features from sequential data consisting of predictable patterns. However, they also tend to be ineffective at representing coherent long-term patterns or understanding non-immediate context, which limits their overall use in composition. Fortunately this shortcoming is partially offset in this scenario by the fact that a skilled human composer will make modifications to the output from this stage as they direct the EA. Because of this it is not necessary for the population generator's output to fully account for certain long-term features, provided that it produces a reasonable variety of good material.
	
	%TODO: Mention use of Gray Code in durations
	
	\subsection{Training}
	
	As discussed in sections \ref{sec:fitness} and \ref{sec:pop-gen}, the AI contains several components using supervised learning. Therefore the results produced by these components, and by extension the AI, will be heavily dependent on the training data provided.
	
	Providing training data for the AI to learn from is potentially a non-trivial task. The reason for this is that the data must fulfil a certain set of constraints, namely:
	
	\begin{itemize}
		\item The data must be stored in a machine-readable format
		\item The input must not contain any features that the AI is not capable of parsing
		\item There must be a strong degree of consistency within the data so that the learned patterns are coherent
	\end{itemize}
	
	The format used to store music throughout this project is the Midi file format \cite{swift1997brief}, which stores music in the form of notes and events instead of sound. This is ideal for this application, as it allows for direct recovery of the exact musical elements contained within each piece of music, a significantly more challenging task when dealing with raw audio. It is also convenient for data acquisition as there exist various large collections of royalty-free midis available to the public. Unfortunately these midis are very often classical music of some variety which often fails to satisfy the second requirement.
	
	The second requirement is one of the most constricting, as for various reasons the AI has many restrictions on the type of musical data it can work with. %TODO: Definitely reference back on this point
	The most harmful of these is the inability to parse or generate polyphonic melodies. This causes some major issues when working with orchestral or piano music due to their frequent use of many melodic lines and chords respectively, which excludes most available classical music. Although it is in some cases possible to isolate a single melodic line, there are quite often no such lines that are both long and substantial enough to be useful for training.
	
	To best overcome these restrictions, the data used was a selection of nursery rhymes transcribed for this research. These have the advantage of being relatively simple and sharing many commonalities such as key and meter, making it much easier for the AI to identify and recreate patterns. They are also universally monophonic due to their vocal nature.
	
	\section{Implementation}
	
		
	\backmatter
	
	\bibliography{dissertation}
	
	\appendix
	
	\chapter{An Example Appendix}
	\label{appx:example}
	
	Content which is not central to, but may enhance the dissertation can be 
	included in one or more appendices; examples include, but are not limited
	to
	
	\begin{itemize}
		\item lengthy mathematical proofs, numerical or graphical results which 
		are summarised in the main body,
		\item sample or example calculations, 
		and
		\item results of user studies or questionnaires.
	\end{itemize}
	
	\noindent
	Note that in line with most research conferences, the marking panel is not
	obliged to read such appendices.
	
	% =============================================================================
	
\end{document}
